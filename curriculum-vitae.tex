\documentclass[paper=a4,fontsize=11pt]{scrartcl} % KOMA-article class

\usepackage[english]{babel}
\usepackage[utf8x]{inputenc}
\usepackage[protrusion=true,expansion=true]{microtype}
\usepackage{amsmath,amsfonts,amsthm}     % Math packages
\usepackage{graphicx}                    % Enable pdflatex
\usepackage[svgnames]{xcolor}            % Colors by their 'svgnames'
\usepackage{geometry}                    % Saving trees
    \headheight=-10px
    \headsep=0px
    \marginparwidth=0px
    \textheight=740px
    \textwidth=460px
    \hoffset=-23px
    \usepackage{url}
    \definecolor{urlcolor}{rgb}{0,0,0.65}
    %\usepackage[colorlinks=true, backref=page, urlcolor=urlcolor]{hyperref}
    \usepackage[hidelinks=true]{hyperref}
\usepackage{tgtermes}
\usepackage{tgbonum}

\frenchspacing              % Better looking spacings after periods
\pagestyle{empty}           % No pagenumbers/headers/footers

%%% Custom sectioning (sectsty package)
%%% ------------------------------------------------------------
\usepackage{sectsty}

%\sectionfont{
%   \usefont{OT1}{phv}{b}{n}%       % bch-b-n: CharterBT-Bold font
%   \sectionrule{0pt}{0pt}{-5pt}{3.5pt}}

\sectionfont{%                      % Change font of \section command
    \usefont{OT1}{qbk}{b}{n}%       % qbk-b-n: TeX Gyre Bonum Bold
    \sectionrule{0pt}{0pt}{-8pt}{3.5pt}}

%%% Macros
%%% ------------------------------------------------------------
\newlength{\spacebox}
\settowidth{\spacebox}{8888888888}          % Box to align text
\newcommand{\sepspace}{\vspace*{1em}}       % Vertical space macro

\definecolor{dark-grey}{gray}{0.15}

% TeX Gyre Bonum bold
\newcommand{\BonumBold}[1]{\usefont{OT1}{qbk}{b}{n} #1}
% TeX Gyre Bonum
\newcommand{\Bonum}[1]{\usefont{OT1}{qbk}{n}{n} #1}
% TeX Gyre Termes
\newcommand{\Termes}[1]{\usefont{OT1}{qtm}{m}{n} #1}

\newcommand{\MyName}[1]{ % Name
        \Huge \BonumBold{\hfill #1}
        \par \normalsize \normalfont}

\newcommand{\MySlogan}[1]{ % Slogan (optional)
        \large \Termes{\hfill \textit{#1}}
        \par \normalsize \normalfont}

\newcommand{\NewPart}[1]{\section*{\lowercase{#1}}}

\newcommand{\DeobfsAddr}[6]{{#1}{#5}{#4}{#3}{#2}{#6}}

\newcommand{\PersonalEntry}[2]{
        \noindent\hangindent=2em\hangafter=0 % Indentation
        \parbox{\spacebox}{        % Box to align text
        %\parbox[t]{3.5cm}{
        \textit{#1}}               % Entry name (address, email, etc.)
        \hspace{2em}{\color{dark-grey}\footnotesize #2 }\par % Entry Value
}

\newcommand{\CorporateEntry}[2]{
        \noindent\hangindent=2em\hangafter=0 % Indentation
        \parbox{\spacebox}{        % Box to align text
        %\parbox[t]{3.5cm}{
        \textit{#1}}               % Entry name (address, email, etc.)
        %\hspace{2em}
               {
          \parbox[t][0.5em]{12.5cm}{
          \noindent\hangindent=2em\hangafter=0{
            \color{dark-grey}
            \footnotesize #2 }}
          \normalsize \par}} % Entry value

\newcommand{\SkillsEntry}[2]{      % Same as \PersonalEntry
        \noindent\hangindent=2em\hangafter=0 % Indentation
        \parbox{\spacebox}{        % Box to align text
        \textit{#1}}               % Entry name
        \parbox[t][2.5em]{12.5cm}{%
          \noindent\hangindent=30px\hangafter=0{%
          \footnotesize #2}}%      %Entry value
        \normalsize \par}

\newcommand{\EducationEntry}[4]{
        \noindent \textbf{#1} \hfill        % Study
        \colorbox{Black}{%
          \makebox(100,10){%
            \color{White}\textbf{#2}}} \par % Duration
        \noindent \textit{#3} \par          % School
        \noindent\hangindent=2em\hangafter=0 {%
          \color{dark-grey}%
          \Bonum{\footnotesize #4}}%        % Description
        \normalsize \par}

\newcommand{\WorkEntry}[4]{                 % Same as \EducationEntry
        \noindent \textbf{#1} \hfill        % Jobname
        \colorbox{Black}{%
          \makebox(100,10){%
            \color{White}\textbf{#2}}} \par % Duration
        \noindent \textit{#3} \par          % Company
        \noindent\hangindent=2em\hangafter=0{%
          \color{dark-grey}%
          \Bonum{\footnotesize #4}}%        % Description
        \normalsize \par}

%%% Begin Document
%%% ------------------------------------------------------------
\begin{document}
% you can upload a photo and include it here...
%\begin{wrapfigure}{l}{0.5\textwidth}
%   \vspace*{-2em}
%       \includegraphics[width=0.15\textwidth]{photo}
%\end{wrapfigure}

\MyName{isis agora lovecruft}
\MySlogan{curriculum vitae}

\sepspace

%%% Personal details
%%% ------------------------------------------------------------
\NewPart{Personal}{}

\PersonalEntry{Github}
  {\href{https://github.com/isislovecruft}{https://github.com/isislovecruft}}
\PersonalEntry{Twitter}
  {\href{https://twitter.com/isislovecruft}{@isislovecruft}}
\PersonalEntry{Email}
  {\href{mailto:\DeobfsAddr{isis}{id.}{nthevo}{nsi}{patter}{net}}
    {\DeobfsAddr{isis}{id.}{nthevo}{nsi}{@patter}{net}}}
\PersonalEntry{Blog}
  {\href{https://fyb.patternsinthevoid.net/}{https://patternsinthevoid.net}}
\PersonalEntry{Pronouns}
  {\href{https://pronoun.is/they/them}{they/them}}

%%% Business and Corporate Stuff
%%% ------------------------------------------------------------
\NewPart{Corporate}{}

I'm the founder and owner of a security and cryptography
contracting and consulting firm, Patterns in the Void, Ltd.  Past
clientele have included The Tor Project, Signal Foundation, and others.
\sepspace

\CorporateEntry{Company}{Patterns in the Void, Ltd.}
\CorporateEntry{Title}{Owner, Founder, and Chief Executive Officer}
\CorporateEntry{Address}{297 Kingsbury Grade \#100, P.O. Box 4470}
\CorporateEntry{}{Lake Tahoe, Nevada 89449 USA}

\sepspace

%%% Work experience
%%% ------------------------------------------------------------
\NewPart{Professional experience}{}

\WorkEntry{Applied Cryptographer}{2018 - present}
 {Private Funder, Research Grant}{Custom elliptic-curve cryptographic
 signature research, design, development, and optimisation.}
\sepspace

\WorkEntry{Applied Cryptographer}{2018}
 {Signal Foundation, Contractor}{Custom zero-knowledge proof design
 and multi-architecture cryptographic development targeting numerous
 mobile devices and a web assembly API for a browser extension.}
\sepspace

\WorkEntry{Applied Cryptographer}{2016 - present} {Dalek Cryptography,
  Contractor}{Research on elliptic curve cryptography, zero-knowledge protocol
  design, and anonymous credentials.  Co-authored what is now the world's
  fastest cryptographic library, \textsc{curve25519-dalek}. See
  \href{https://github.com/dalek-cryptography}{the Dalek Cryptography Github
    organisation}.}
\sepspace

\WorkEntry{Core Tor Software Developer}{2010 -- 2018}{The Tor Project,
  Contractor}{Lead developer and maintainer of systems and components for Tor
  bridge distribution, which includes managing access to a database of
  all secret entrance relays in the Tor network, compartmentalising
  this data via hashring structures in order to restrict access by
  which distribution method is used, and securing the interfaces by
  which which Tor users may retrieve this data from adversaries
  hostile to Tor usage.  Past work on identifying and patching user
  fingerprinting and security issues
  for Tor Browser.  Ongoing efforts include improving Tor's circuit-level protocols
  and cryptography, designing and implementing an
  anonymous-credentials based system for censorship-resilient secret
  sharing, and reviewing RFC-like proposals for changes to the Tor protocol. }
\sepspace

\WorkEntry{Software Developer and Security Consultant}{2011 -- 2012}{LEAP
  Encryption Access Project, Part-time}{Development of several asynchronous
  servers, including a distributed and scalable transparently-encrypting
  remailer, which, at the time of design and implementation, handled seventeen
  million email users per day.  The system, without modification, now handles
  over a hundred million daily users.  Conducted security audits for systems
  components and dependencies, and filed several CVEs in widely used
  software, from Python's package manager to a GnuPG library in use by
  over thirty Bitcoin exchanges. Provided extensive consulting with respect
  to systems architecture and cryptographic engineering, including the
  design of an alternative OpenPGP keyserver.}
\sepspace

\WorkEntry{Software Design Consultant}{2011 -- 2012}{Electronic Frontier
  Foundation}{Worked with developers from the Electronic Frontier Foundation's
  (EFF) technical team to design a system for heuristic classification of
  benign network anomalies (e.g. due to NAT routers mangling packets) and
  malicious behaviours in TCP/UDP packet routing.  The system, called
  Switzerland, can be used between two parties to determine if a malicious
  adversary is altering their digital communications, as well as to inform
  users of the types of malicious behaviours present in packet flows.}
\sepspace

\WorkEntry{Software Developer and Distributed Systems Architect}{2010 -- 2012}
{The Tor Project/Open Observatory of Network Interference}{Reverse
  engineered the software and network testing methodologies of dozens of
  proprietary network testing and anomaly detection software tools.  Designed
  the Open Observatory of Network Inference (OONI): a global, distributed
  platform for detection and measurement of network anomalies, including
  online censorship and both passive and active malicious behaviours.
  Designed and developed distributed backend systems for collection of
  measurement data which are now deployed on over 300,000 MLab and PlanetLab
  servers worldwide.  Researched, designed, and published open specifications
  for effective methodologies for detection of anomalous behaviour.  Designed
  and developed numerous tests which run on the OONI platform, including tests
  for misbehaving and poisoned DNS servers, captive portal detection and
  misbehaviour, and anomalous behaviours within Transport Layer Protocol (TLS)
  handshake and session resumption negotiation.}
\sepspace

\WorkEntry{Cryptographic Protocol Researcher}{2009 -- 2012}{Open
  Whispersystems}{Designed a modified Fully-Hashed Menezes-Qu-Vanstone
  protocol in an attempt to design a multi-party, forward-secret,
  end-to-end-encrypted, synchronous communications protocol with low overhead,
  such that it could feasibly be deployed over an SMS-based transport within
  Open Whispersystems flagship open source product TextSecure (now called
  Signal).  Later, researched the feasibility and efficiency of applying the
  Multi-Party Off-The-Record (MPOTR) protocol.  During this time, I also
  mentored several students in various secure mobile applications development
  projects, including aspects of protocol design, cryptography, and backend
  systems design, for Open Whispersystems.}
\sepspace

\newpage

\WorkEntry{Machine Learning Researcher}{2008 -- 2010}{Private Grant}
  {Ported a fully back-propagational linguistic neural network from SPARC
  to x86 and made several optimisations allowing for parallelisation,
  developed an extensive interface for researchers to obtain data
  regarding neural states during execution, and constructed an OpenMPI-based
  cluster on which several research experiments on the neural network were
  conducted over the following two years.}
\sepspace

\WorkEntry{Security and Cryptographic Design Consultant}{2007 -- present}
{Private Clientele, Contractor}{Security auditing and cryptographic
  design consulting to numerous clients, including several startups within the
  Bitcoin community.  Conducted security audits for several S\&P 500 companies
  to find exploitable vulnerabilities in networked applications, including
  banking and financial software, hypervisor and virtual machine management
  software, and a major browser.}
\sepspace

\WorkEntry{Software Developer}{2006 -- 2009}{March Hare Communications
  Collective}{Designed and developed open source mobile applications for
  grass-roots activists to conduct better crisis management and have
  capabilities for secure communication in political protest situations.  The
  development was centered around the Android platform, but work was conducted
  alongside an iOS developer in order to produce a cross-platform solution.
  The applications included solutions for secure crisis mapping in hostile
  situations, secure location and event reporting, secure and metadata-free
  synchronous messaging, and a tool to remotely trigger the deletion of
  personal information on a device.}
\sepspace

\WorkEntry{Various Contributions}{2006 -- present}{Volunteer}{Volunteer
  contributions to numerous Open Source Software (OSS) projects,
  including Open Whispersystems, March-Hare Communications Collective, LEAP
  Encryption Access Project, Briar Project, Tahoe-LAFS, The Tor Project, the
  Electronic Frontier Foundation, and others.}
\sepspace

\NewPart{Mentorships \& Leadership Roles}{}

\WorkEntry{Tor Internship}{2017}{Mentor}{Conducted a hiring and
  interviewing process, and then lead an intern in the
  development of several Rust libraries and tools for measuring the
  available bandwidth of Tor bridge relays.}
\sepspace

\WorkEntry{Tor Summer of Privacy}{2015}{Assistant Mentor}{Assisted in the
  mentorship of a volunteer student project, called GetTor, which provides
  alternate mechanisms for securely downloading and installing Tor and/or Tor
  Browser in places where access to The Tor Project servers is censored.}
\sepspace

\WorkEntry{Google Summer of Code}{2014}{Mentor}{Mentored a student in
  designing a new distribution system for Tor bridge relays to Tor users in
  censored regains via Twitter's HTTP API.}
\sepspace

\WorkEntry{Google Summer of Code}{2013}{Assistant Mentor}{Mentored a student project to
  design and implement a censorship analysis system \\
  (\href{https://explorer.ooni.torproject.org/world/}
  {https://explorer.ooni.torproject.org/world/}) using data from the Open
  Observatory of Network Interference.}
\sepspace

%\newpage

\WorkEntry{Open Whispersystems Spring Break of Code}{2012}{Mentor}{Mentored
  several students in asynchronous programming and scalable redesign of
  several backend servers for an encrypted voice call system, secure location
  sharing, and server-private contact list storage.}
\sepspace

\NewPart{Invited Talks}{}

%%% XXX maybe fill in this section, not sure. i don't want to hilight my talks

Slides from my talks including their LaTeX sources are generally made
available in \href{https://github.com/isislovecruft/talks}{a public
Github repo}, and video--when available--is provided on
\href{https://www.youtube.com/channel/UClJDPyhdnSkGoqb-x9gImhA}{my YouTube channel}.

\WorkEntry{Rustconf}{2018}{Portland, Oregon}{Co-presented a talk on
  incrementally rewriting the Tor network daemon in Rust and the security
  vulnerabilities and challenges we found and faced along the way.}
\sepspace

\WorkEntry{Noisebridge}{2017}{San Francisco, California}{Presented a history of
  security vulnerabilities and side-channel attacks on real-world cryptographic
  libraries, along with the countermeasures we took to avoid these problems in
  the design of the Dalek Cryptography libraries.}
\sepspace

\WorkEntry{Rustconf}{2017}{Portland, Oregon}{Co-presented a talk on
  creating curve25519-dalek, which is currently the world's fastest cryptographic
  library and is in use by billions of people.}
\sepspace

\WorkEntry{University of Waterloo}{2016}{Waterloo, Ontario, Canada}{Invited talk
  presenting Hyphae, \href{https://patternsinthevoid.net/hyphae}, a
  censorship-resistant system for using zero-knowledge proof-of-social-graphs
  and private behaviour tracking via anonymous credentials based on algebraic
  MACs, to the Cryptography, Security, and Privacy (CrySP) department.}
\sepspace

\WorkEntry{Radboud Universiteit}{2016}{Nijmegen, Netherlands}{Guest
  lecture for the Advanced Network Security course on the history and current
  status of mixnets, anonymity networks, and anonymous communcations systems,
  given at Raboud Unversiteit in Nijmegen to students following the
  \href{https://www.true-security.nl/}{TRU/e Computer Security Master's
    Programme}.  Afterwards, I freely posted the
  \href{https://www.youtube.com/watch?v=xGIE7KTJiBY&feature=youtu.be}
    {video recording on YouTube}, where it received 4,000 views within the
  first week.}
\sepspace

\WorkEntry{Radboud Universiteit}{2016}{Nijmegen, Netherlands}{Talk
  given to graduate students, researchers, and faculty of the Digital
  Security group at the \href{http://www.ru.nl/icis/}{Institute for Computing and
  Information Sciences} at Raboud Unversiteit
  in Nijmegen, the Netherlands, concerning my work with
  \href{https://www.torproject.org/}{The Tor Project} on protecting Tor bridges
  from discovery by nation-state adversaries.}
\sepspace

\WorkEntry{EI/$\psi$}{2015}{Utrecht, Netherlands}{I
  \href{https://www.win.tue.nl/eipsi/cwg/aank\%20dec\%202015.pdf}{spoke} to
  the \href{https://www.win.tue.nl/eipsi/seminars_cwg.html}{Cryptography
  Working Group} at the \href{http://www.win.tue.nl/eipsi}{Eindhoven
  Institute for the Protection of Systems and Information} (part of
  Wiskunden en Informatica at Technisches Universiteit Eindhoven) in Utrecht,
  the Netherlands, about the cryptography for my current work on using
  pairing-based anonymous credential schemes for social distribution of Tor
  bridge relays.}
\sepspace

\WorkEntry{ThoughtWorks}{2015}{Berlin, Germany}{
  I spoke about Tor's circuits, path selection, and hidden services, (and the
  basics of the cryptography for those things) at a women's-only
  \href{http://www.meetup.com/de/ThoughtWorks-Technology-Radar-Briefing-Berlin/events/226058216/}
  {event held at ThoughtWorks' Werkstatt Berlin} space.}
\sepspace

\WorkEntry{FOSDEM: FOSS Developers' European Meeting}{2013}{Brussels,
  Belgium}{I spoke on mine and my team's work to create a peer-reviewed
  taxonomy for discussion of surveillance and censorship, as well as a free
  and open source software toolset, called the Open Observatory of Network
  Interference (OONI), for producing open data.  This talk covered
  the vision of OONI as well as technical details of our tools. (See
  \href{https://archive.fosdem.org/2013/schedule/event/ooni/}
       {https://archive.fosdem.org/2013/schedule/event/ooni/} for
       recordings.)}
\sepspace

\WorkEntry{DebConf'13: Debian Developers Conference}{2013}{Vaumarcus,
  Switzerland}{Workshop with Debian Developers on tactics for higher-security
  package management, including an introduction to software reproducible builds for
  Debian package maintainers.}
\sepspace

%\newpage

\WorkEntry{Google}{2012}{Brussels, Belgium}{Talk to Google's Internal Security
  Team and the Measurement Lab (MLAB) team regarding secure distributed
  systems design and measurement data protection and integrity, with respect
  to the deployment of a global distributed network anomaly detection system
  on 300,000 machines.}
\sepspace

\WorkEntry{University of Washington, Olympia}{2009}{Olympia,
  Washington}{Lecture and workshop for journalism students in the Department
  of Communication at the University of Washington, on the topic of secure
  communications for journalists and their sources.}
\sepspace

%%% Skills
%%% ------------------------------------------------------------
\NewPart{Skills}{}

%% In order of how willing I am to write them:
\SkillsEntry{Languages}{\textsc{Rust}, \textsc{C/C++},
  \textsc{Python}, \textsc{Swift}, \textsc{Julia},
  \textsc{Go}, \textsc{x86/ARM/MIPS ASMs}, \textsc{Common Lisp}, \textsc{Bash}, \textsc{Lua},
  \textsc{Javascript}, HTML5, CSS3, and \textsc{Java}}

%% In no particular order and clearly incomplete because it says nothing about
%% bicycle mechanics, asciiart, or bytebytes:
\SkillsEntry{Software}{Low- to High-Level Cryptographic Design, Cryptographic Engineering,
  Network Programming, Asynchronous Programming, Distributed Systems Design,
  Misuse-Resistant API Design, Security Best Practices}

%\noindent\hangindent=2em\hangafter=0
%\parbox{\spacebox}{
%  \textit{Bytebeats}}
%\parbox[t][2.5em]{12.5cm}{
%  \noindent\hangindent=30px\hangafter=0
%    \small
%\begin{verbatim}
%
%python3 -c'print(foo)'
%
%\end{verbatim}
%\normalsize\par}

%x='if(t%2)else';python3 -c'[print(t>>15&(t>>(2$x 4))%(3+(t>>(8$x 11))%4)+\
%(t>>10)|42&t>>7&t<<9,end='')for t in range(2**20)]'|aplay -c2 -r4
%

%\SkillsEntry{Bytebeats}{
%  \begin{minipage}{\textwidth}
    
%\end{minipage}}

%%
%% x=''if(t%2)else'';python3 -c''[print(t>>15&(t>>(2$x 4))%(3+(t>>(8$x 11))%4)\
%%   
%%
%%                   ↑↑ CLEARLY MY GREATEST SKILL ↑↑

%%% References
%%% ------------------------------------------------------------
\NewPart{References}{}
\hspace{0.6cm} \textit{Available upon request}

%XXX: You could do name-dropping here, but I'm not sure whether that's helpful
%     or required.  For example, listing Ian, who would probably be willing to
%     write you a letter that he himself wanted to have you as a student
%     looks quite impressive... not sure whether it's helpful or necessary,
%     though.

\end{document}

%%%%%%%%%%%%%%%%%%%%%%%%%%%%%%%%%%%%%%%%%%%%%%%%%%%%%%%%%%%%%%%%%%%%%%
% LaTeX Template: Curriculum Vitae
%
% Source: http://www.howtotex.com/
% Feel free to distribute this template, but please keep the
% referal to HowToTeX.com.
% Date: July 2011
%
%%%%%%%%%%%%%%%%%%%%%%%%%%%%%%%%%%%%%%%%%%%%%%%%%%%%%%%%%%%%%%%%%%%%%%
\documentclass[paper=a4,fontsize=11pt]{scrartcl} % KOMA-article class

\usepackage[english]{babel}
\usepackage[utf8x]{inputenc}
\usepackage[protrusion=true,expansion=true]{microtype}
\usepackage{amsmath,amsfonts,amsthm}     % Math packages
\usepackage{graphicx}                    % Enable pdflatex
\usepackage[svgnames]{xcolor}            % Colors by their 'svgnames'
\usepackage{geometry}                    % Saving trees
    \headheight=-10px
    \headsep=0px
    \marginparwidth=0px
    \textheight=740px
    \textwidth=460px
    \hoffset=-23px
\usepackage{url}
\usepackage[hidelinks]{hyperref}
\usepackage{tgtermes}
\usepackage{tgbonum}

\frenchspacing              % Better looking spacings after periods
\pagestyle{empty}           % No pagenumbers/headers/footers

%%% Custom sectioning (sectsty package)
%%% ------------------------------------------------------------
\usepackage{sectsty}

%\sectionfont{
%   \usefont{OT1}{phv}{b}{n}%       % bch-b-n: CharterBT-Bold font
%   \sectionrule{0pt}{0pt}{-5pt}{3.5pt}}

\sectionfont{%                      % Change font of \section command
    \usefont{OT1}{qbk}{b}{n}%       % qbk-b-n: TeX Gyre Bonum Bold
    \sectionrule{0pt}{0pt}{-8pt}{3.5pt}}

%%% Macros
%%% ------------------------------------------------------------
\newlength{\spacebox}
\settowidth{\spacebox}{8888888888}          % Box to align text
\newcommand{\sepspace}{\vspace*{1em}}       % Vertical space macro

\definecolor{dark-grey}{gray}{0.15}

% TeX Gyre Bonum bold
\newcommand{\BonumBold}[1]{\usefont{OT1}{qbk}{b}{n} #1}
% TeX Gyre Bonum
\newcommand{\Bonum}[1]{\usefont{OT1}{qbk}{n}{n} #1}
% TeX Gyre Termes
\newcommand{\Termes}[1]{\usefont{OT1}{qtm}{m}{n} #1}

\newcommand{\MyName}[1]{ % Name
        \Huge \BonumBold{\hfill #1}
        \par \normalsize \normalfont}

\newcommand{\MySlogan}[1]{ % Slogan (optional)
        \large \Termes{\hfill \textit{#1}}
        \par \normalsize \normalfont}

\newcommand{\NewPart}[1]{\section*{\lowercase{#1}}}

\newcommand{\DeobfsAddr}[6]{{#1}{#5}{#4}{#3}{#2}{#6}}

\newcommand{\PersonalEntry}[2]{
        \noindent\hangindent=2em\hangafter=0 % Indentation
        \parbox{\spacebox}{        % Box to align text
        \textit{#1}}               % Entry name (address, email, etc.)
        \hspace{2em}{\color{dark-grey}\footnotesize #2 }\par} % Entry value

\newcommand{\SkillsEntry}[2]{      % Same as \PersonalEntry
        \noindent\hangindent=2em\hangafter=0 % Indentation
        \parbox{\spacebox}{        % Box to align text
        \textit{#1}}               % Entry name
        \parbox[t][2.5em]{12.5cm}{%
          \noindent\hangindent=30px\hangafter=0{%
          \footnotesize #2}}%      %Entry value
        \normalsize \par}

\newcommand{\EducationEntry}[4]{
        \noindent \textbf{#1} \hfill        % Study
        \colorbox{Black}{%
          \makebox(100,10){%
            \color{White}\textbf{#2}}} \par % Duration
        \noindent \textit{#3}}% \par          % School
%        \noindent\hangindent=2em\hangafter=0 {%
%          \color{dark-grey}%
%          \Bonum{\footnotesize #4}}%        % Description
%        \normalsize \par}


\newcommand{\WorkEntry}[4]{                 % Same as \EducationEntry
        \noindent \textbf{#1} \hfill        % Jobname
        \colorbox{Black}{%
          \makebox(100,10){%
            \color{White}\textbf{#2}}} \par % Duration
        \noindent \textit{#3} \par          % Company
        \noindent\hangindent=2em\hangafter=0{%
          \color{dark-grey}%
          \Bonum{\footnotesize #4}}%        % Description
        \normalsize \par}

%%% Begin Document
%%% ------------------------------------------------------------
\begin{document}
% you can upload a photo and include it here...
%\begin{wrapfigure}{l}{0.5\textwidth}
%   \vspace*{-2em}
%       \includegraphics[width=0.15\textwidth]{photo}
%\end{wrapfigure}

\MyName{isis agora lovecruft}
\MySlogan{curriculum vitae}

%%% Personal details
%%% ------------------------------------------------------------
\NewPart{Personalangaben}{}

%\PersonalEntry{Twitter}
%  {\href{https://twitter.com/isislovecruft}{@isislovecruft}}
\PersonalEntry{Github}
  {\href{https://github.com/isislovecruft}{https://github.com/isislovecruft}}
\PersonalEntry{Email}
  {\href{mailto:\DeobfsAddr{isis}{id.}{nthevo}{nsi}{patter}{net}}
    {\DeobfsAddr{isis}{id.}{nthevo}{nsi}{@patter}{net}}}
\PersonalEntry{OpenPGP}
  {\href{https://blog.patternsinthevoid.net/isis.txt}
  {0A6A58A14B5946ABDE18E207A3ADB67A2CDB8B35}}

%%% Education
%%% ------------------------------------------------------------
%\NewPart{Bildung}{}
%

%\EducationEntry{MSc. Name of Education}{2010-2012}{Name of
%  University}{Descriptive text goes here. In order to maintain a stylish look,
%  try to fill this description with a few lines of text. Do the same for the
%  other entries in the education section.}
%\sepspace
%
%\EducationEntry{BSc. Name of Education}{2007-2010}{Name of
%  University}{Descriptive text goes here. In order to maintain a stylish look,
%  try to fill this description with a few lines of text. Do the same for the
%  other entries in the education section.}

%%% Work experience
%%% ------------------------------------------------------------
\NewPart{Berufserfahrung}{}

\WorkEntry{Verschiedenen Beiträgen}{2006 – jetzt}{Voluntärin}{Freiwillige
  Beiträge bei zahreiche Open Source Software (OSS) Projekte,
  einschließlich Open Whispersystems, March-Hare Communications Collective, LEAP
  Encryption Access Project, Briar Project, Tahoe-LAFS, The Tor Project, the
  Electronic Frontier Foundation, und andere.}
\sepspace

\WorkEntry{Softwareentwicklerin}{2010 – 2012}{The Tor Project,
  unabhängige Auftragnehmerin}{Reverse-Engineering, Software-Design,
  und Softwareentwicklung für das Offen Observatorium der Netzwerkinterferenzen
  (original: Open Observatory of Network Interference) (OONI), ein plattform für
  Erkennung der verteilten, weltweiten, digitalen Zensur und Messungtechnik zu bauen.}
\sepspace

\WorkEntry{Softwareentwicklerin und Computersicherheitsberuflerin}{2011 – 2012}{LEAP
  Encryption Access Project, Teilzeit}{Entwicklung der mehreren asynchronen
  Server, einschließlich ein transparent verschlüsselnd Remailer, Sicherheitsaudit
  für Komponenten und Abhängigkeiten, und Beratung bezüglich Systemarchitektur
  und Kryptographischesenwicklung.}
\sepspace

\WorkEntry{Sicherheits- und Kryptographische Entwicklungs- Beraterin}{2012 – jetzt}
 {Privatkundschaft, Auftragnehmerin}{Sicherheitsaudit und kryptographische
 Entwicklungs Beratung bei verschieden OSS Projekte, einschließlich kryptographische
 Libraries, die bei das Bitcoin Gemeinschaft benutzt.}
\sepspace

\WorkEntry{Hauptsoftwareentwicklerin}{2013 – jetzt}{The Tor Project,
  unabhängige Auftragnehmerin}{Hauptentwickerin und Erhalterin von System und Komponenten
  der Tor Netzwerk Brückenvertrieb.}


%%% Skills
%%% ------------------------------------------------------------
\NewPart{Faehigkeiten}{}

%% In order of how willing I am to write them:
\SkillsEntry{Sprachen}{\textsc{C/C++}, \textsc{Python},
  \textsc{Rust}, \textsc{Go},
  \textsc{x86 ASM}, \textsc{Common Lisp}, \textsc{Bash}, \textsc{Lua},
  \textsc{Javascript}, HTML5, CSS3, and \textsc{Java}}

%% In no particular order and clearly incomplete because it says nothing about
%% bicycle mechanics, asciiart, or bytebytes:
\SkillsEntry{Software}{Netwerkprogrammieren, Asynchronprogrammieren,
  verteilte Systeme Entwicklung, kryptographische Engineering, API Design,
  bewährte Verfahren der Sicherheits}
%%
%% x=''if(t%2)else'';python3 -c''[print(t>>15&(t>>(2$x 4))%(3+(t>>(8$x 11))%4)\
%%   +(t>>10)|42&t>>7&t<<9,end='')for t in range(2**20)]''|aplay -c2 -r4
%%
%%                   ↑↑ CLEARLY MY GREATEST SKILL ↑↑

%%% References
%%% ------------------------------------------------------------
\NewPart{Referenzen}{}
\hspace{0.6cm} \textit{sind auf Anfrage erhältliche}
\end{document}
